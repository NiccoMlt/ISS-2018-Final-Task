%% Direttive TeXworks:
% !TeX root = ../report.tex
% !TEX encoding = UTF-8 Unicode
% !TEX program = arara
% !TEX TS-program = arara
% !TeX spellcheck = it-IT

% arara: pdflatex: { synctex: yes, shell: yes, interaction: nonstopmode }
% arara: pdflatex: { synctex: yes, shell: yes, interaction: nonstopmode }

Per realizzare il task finale proposto, la software house si propone di utilizzare una metodologia agile \textit{Scrum-like}
seguendo un approccio \textit{top-down} \textit{model-driven};
il lavoro sarà dunque diviso in Sprint successivi al termine di ognuno dei quali verrà compiuta una \textit{Sprint Review}
con il committente, il quale potrà suggerire correzioni e nuovi requisiti.
Di conseguenza, al termine di ogni Sprint l'obiettivo sarà di consegnare al cliente un prodotto di valore (\textit{valuable}),
per quanto non completo in tutte le sue parti.

In ogni Sprint verranno eseguite tutte le fasi di un processo di sviluppo software, affinando progressivamente il risultato,
migliorando il prodotto esistente e affrontando le problematiche non ancora analizzate;
lo sviluppo procederà dunque con un approccio incrementale e un'andatura monotona crescente, promuovendo da un lato
l'\textit{estendibilità} e il \textit{cambiamento}, senza però impattare pesantemente il prodotto già consegnato.

Per ogni Sprint, dunque, si procederà per fasi:
la prima fase sarà un'attenta \textbf{analisi dei requisiti}, con l'obiettivo di di identificare con chiarezza
tutti i requisiti del sistema, tracciandoli in maniera univoca e identificando le entità in gioco.
Successivamente, è importante effettuare un'\textbf{analisi del problema}, nella quale verranno sviscerate tutte le problematiche
relative alla progettazione del sistema (senza però scendere in dettagli implementativi),
rimanendo \textit{technology-independent} pur essendo \textit{technology-aware}.
Definita un'architettura logica, si procede con la \textbf{progettazione} e con il \textbf{test planning}.
