%% Direttive TeXworks:
% !TeX root = ../report.tex
% !TEX encoding = UTF-8 Unicode
% !TeX spellcheck = it-IT

\appendix
  \section{Immagine del Raspberry Pi}\label{app:raspi}

  La software house metteva a disposizione una immagine della distribuzione Raspbian\footnote{\url{https://www.raspberrypi.org/downloads/raspbian/}}
  (una versione di Debian per Raspberry Pi) adattata dalla software house per l'uso interno.

  L'immagine è però basata su una vecchia versione di Raspbian Jessie datata 26 febbraio 2016 e presenta diversi problemi.
  Già in passato la software house aveva riscontrato problemi e aveva proposto \textit{fix}
  documentati sul sito\footnote{\url{http://htmlpreview.github.io/?https://github.com/anatali/iss2018/blob/master/it.unibo.issMaterial/issdocs/Material/LectureCesena1819.html}}
  della software house.

  La configurazione standard prevede l'accesso alla rete pubblica tramite WiFi e la configurazione locale tramite connessione diretta ethernet;
  poiché la macchina di sviluppo utilizzata non dispone di connessione ethernet, si è deciso,
  consci anche degli altri problemi presenti nell'immagine, di adattare una versione più moderna di Raspbian Buster lite per questo progetto.

  \subsection{Configurazione del Raspberry Pi}\label{app:raspi:conf}

  Si è scelto di utilizzare un modulo WiFi USB in aggiunta a quello integrato nel Raspberry Pi 3,
  ma è possibile utilizzare il solo modulo WiFi integrato sia come access point che come client, pur con maggiori instabilità e minori velocità.

  La procedura di adattamento dell'immagine è stata documentata in modo da poter essere riproducibile in futuro.

  \begin{enumerate}
    \item
      La base dell'immagine è l'ultima versione di \textbf{Raspbian} nella sua versione minimale \textbf{lite}:
      in questo caso, la versione Buster del 26 settembre 2019;
    \item
      l'immagine è stata connessa alla rete tramite WiFi e i pacchetti sono stati aggiornati tramite \texttt{apt};
    \item
      seguendo la documentazione fornita sul sito ufficiale\footnote{\url{https://raspap.com/}},
      è stato installato \textbf{RaspAP} e configurato per funzionare esclusivamente via WiFi:
      \begin{enumerate}
        \item
          \texttt{curl -sL https://install.raspap.com | bash} permette di installare RaspAP\@;
        \item
          seguendo le FAQ\footnote{\url{https://github.com/billz/raspap-webgui/wiki/FAQs\#can-i-use-wlan0-and-wlan1-rather-than-eth0-for-my-ap}}
          al file \texttt{includes/config.php} si è aggiunta \texttt{wlan1} come client interface e si è impostato nel file \texttt{/etc/dhcpcd.conf}
          \texttt{wlan0} come access point
      \end{enumerate}
      In questo modo, il dispositivo dovrebbe essere in grado di collegarsi alla rete WiFi pre-configurata e mettere a disposizione una rete con SSID ``raspi-webgui''.
    \item
      utilizzando lo script ufficiale\footnote{\url{https://github.com/docker/docker-install}}
      (ma si sarebbe potuta seguire la documentazione\footnote{\url{https://docs.docker.com/install/linux/docker-ce/debian/}}), si è installato Docker;
    \item
      si è installato Java 1.8 tramite pacchetto \texttt{openjdk-8-jdk-headless};
    \item
      il broker MQTT Mosquitto viene eseguito tramite container Docker\footnote{\url{https://hub.docker.com/_/eclipse-mosquitto}}.
  \end{enumerate}
