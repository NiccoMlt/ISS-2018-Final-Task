%% Direttive TeXworks:
% !TeX root = ../report.tex
% !TEX encoding = UTF-8 Unicode
% !TEX program = arara
% !TEX TS-program = arara
% !TeX spellcheck = it-IT

% arara: pdflatex: { synctex: yes, shell: yes }
% arara: pdflatex: { synctex: yes, shell: yes }

Essendo lo Sprint iniziale, è stata realizzata un'attenta analisi di tutti i requisiti forniti dal committente
con uno sguardo generale su tutto il sistema, in modo da poter chiarire con quest'ultimo eventuali incomprensioni.

L'obiettivo di questo primo Sprint è individuare struttura e interazioni di massima del sistema; in particolare:
\begin{itemize}
  \item
    con l'analisi dei requisiti si vuole far emergere i componenti del sistema del committente;
  \item
    con l'analisi del problema si vuole identificare eventuali ulteriori componenti di interesse per la futura progettazione.
\end{itemize}

Una volta formalizzata un prima analisi dei requisiti generale, reperibile in questo documento alla \Cref{sec:req_analysis},
il passo successivo è individuare i requisiti prioritari da realizzare in questo primo Sprint,
in modo da poterli analizzare nel dettaglio.

\subsection{Analisi dei requisiti}\labelssec{sp1:req_analysis}

Dopo aver delineato tutti i requisiti base, si è scelto di approfondire quelli maggiormente fondanti e di valore per il prodotto.
