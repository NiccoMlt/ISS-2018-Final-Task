%% Direttive TeXworks:
% !TeX root = ../report.tex
% !TEX encoding = UTF-8 Unicode
% !TEX program = arara
% !TEX TS-program = arara
% !TeX spellcheck = it-IT

% arara: pdflatex: { synctex: yes, shell: yes }
% arara: pdflatex: { synctex: yes, shell: yes }
Dall'analisi dei requisiti forniti dal committente è possibile definire i seguenti requisiti:

\begin{enumerate}
  \item
    il sistema è distribuito e composto da tre componenti principali:
    \begin{description}
      \item[\textbf{Robot \requirementref{discovery}}] Si intende un componente che esplori (\requirementref{R-explore}) autonomamente l'ambiente seguendo specifici comandi dell'operatore;
      \item[\textbf{Robot \requirementref{retriever}}] Si intende un componente che raggiunga la bomba scoperta e la porti a casa;
      \item[\textbf{Componente \requirementref{console}}] Il componente utilizzato da remoto dall'operatore attraverso un PC od uno smartphone;
    \end{description}

  \item
    gli \textbf{ostacoli} sono \textit{statici} e, dato che viene assunto che la hall sia stata evacuata, possono essere divisi in:
    \begin{itemize}
      \item \textit{muri},
      \item \textit{borse}.
    \end{itemize}

  \item
    il \textbf{robot \requirementref{discovery}} deve iniziare l'esplorazione (\requirementref{R-explore}) solo nel caso in cui la temperatura sia sotto la soglia di 25°C (\requirementref{R-TempOk}) e riceva il comando di esplorazione dalla \textbf{\requirementref{console}} (\requirementref{R-startExplore});

  \item
    il comando \textit{Esplora} (\requirementref{R-startExplore}) viene inviato dalla \textbf{\requirementref{console}} unicamente al \textbf{robot \requirementref{discovery}};

  \item
    le informazioni dello stato del \textbf{robot \requirementref{discovery}} vengono inviate dallo stesso alla \textbf{\requirementref{console}} (\requirementref{R-consoleUpdate});
    \begin{description}
      \item[\textbf{Stato del robot}] Si intendono almeno le sue coordinate e l'azione corrente;
    \end{description}

  \item
    il \textbf{robot \requirementref{discovery}} è equipaggiato con un \textbf{sonar} nella sua parte anteriore per permettergli di individuare eventuali ostacoli (\requirementref{R-stopAtBag}) e di un \textbf{led} che dovrà saper far lampeggiare (\requirementref{R-blinkLed});

  \item
    qualora il \textbf{robot \requirementref{discovery}} incontri un ostacolo, deve:
    \begin{itemize}
      \item fermarsi (\requirementref{R-stopAtBag});
      \item scattare una foto (\requirementref{R-takePhoto});
      \item mandare la foto (\requirementref{R-sendPhoto}) alla \textbf{\requirementref{console}}.
    \end{itemize}
    A sua volta, la \textbf{\requirementref{console}} dovrà:
    \begin{itemize}
      \item allertare l'operatore (\requirementref{R-alert});
      \item salvare la foto (\requirementref{R-storePhoto}) ricevuta con dati di contesto;
      \item mandare il robot a casa (\requirementref{R-backHomeSinceBomb}) o farlo continuare (\requirementref{R-continueExploreAfterPhoto}).
    \end{itemize}

  \item
    il \textbf{robot \requirementref{retriever}} deve conoscere il percorso per raggiungere la bomba (\requirementref{R-reachBag}) una volta che il \textbf{robot \requirementref{discovery}} sarà tornato (\requirementref{R-waitForHome}) e il percorso per tornare con la bomba (\requirementref{R-bagAtHome});

  \item
    il cuore del sistema è la definizione della \textit{business logic} del robot che concerne l'esplorazione autonoma della hall;

  \item
    in qualsiasi momento, anche mentre in esplorazione, il \textbf{robot \requirementref{discovery}} deve essere reattivo ad un comando di halt (\requirementref{R-stopExplore}) inviatogli dalla \textbf{\requirementref{console}}.
\end{enumerate}

Nelle sezioni successive ciascun requisito ed entità verrà analizzato con maggior dettaglio.

\subsection{Componenti identificati}\labelssec{components}

\subsubsection{Console}\labelsssec{console}

La console consiste in un'interfaccia grafica attraverso la quale l'operatore può interagire con il sistema.

Essa permette di interagire con il robot esploratore inviandogli un comando per far partire l'esplorazione.
Attraverso la console l'utente può stoppare l'esplorazione per poi scegliere di continuarla oppure di farlo tornare alla base.
In caso di rilevamento di un ordigno la console deve segnalarlo all'utente, salvare la foto ed inviare al robot il comando per tornare alla posizione iniziale.
Infine, deve consentire l'invio di un comando per far sì che un altro robot possa recuperare l'ordigno.

\subsubsection{Robot Discovery}\labelsssec{discovery}

Robot Discovery è un ddr robot che si occupa dell'esplorazione dell'ambiente per rilevare la presenza di eventuali ordigni.
Il robot per il committente è un dispositivo fisico in grado di muoversi nell'ambiente grazie a delle ruote motrici, fornito di un LED
e dotato di un sonar posto frontalmente per individuare ostacoli.

Esso permette di essere avviato e/o stoppato remotamente.
Deve far lampeggiare un LED ed inviare informazione sullo stato suo e dell'esplorazione.
In prossimità di una valigia deve fermarsi, scattare una foto ed inviarla all'operatore.

\subsubsection{Robot Retriever}\labelsssec{retriever}

Robot Discovery è un drr robot che ha il compito di raggiungere la borsa individuata (contenente la bomba) e trasportarla alla posizione iniziale.

\subsection{Dettaglio sui requisiti della fase di esplorazione}\labelssec{explore}

\begin{description}[itemsep=1em]
  \item[\requirementref{R-explore}]\labelsssec{R-explore}

  Durante la fase di esplorazione il robot deve muoversi nella stanza, raccogliendo informazioni per creare una rappresentazione del territorio comprendente ostacoli e posizioni degli ordigni.
  L'output di questa fase consiste in una rappresentazione che deve riportare le informazioni riguardanti la stanza e deve essere condivisa con gli altri componenti del sistema.
  In questa rappresentazione dovranno essere segnalate le posizioni di eventuali ostacoli, in modo che il robot Retriever possa ottimizzare il proprio percorso verso l'ordigno da rimuovere.

  \item[\requirementref{R-startExplore}]\labelsssec{R-startExplore}

  L'esplorazione deve partire in seguito all'invio di un comando da parte dell'operatore.

  \item[\requirementref{R-TempOk}]\labelsssec{R-TempOk}

  L'esplorazione deve partire solo se la temperatura rispetta le condizioni prestabilite.

  \item[\requirementref{R-stopExplore}]\labelsssec{R-stopExplore}

  L'esplorazione deve stopparsi in caso l'utente lo richieda.
  Emerge quindi la necessità da parte del robot di essere reattivo ai comandi ricevuti.

  \item[\requirementref{R-backHome}]\labelsssec{R-backHome}

  Il robot deve tornare alla posizione iniziale se richiesto dall'utente al seguito dello stop dell'esplorazione.

  \item[\requirementref{R-continueExplore}]\labelsssec{R-continueExplore}

  Il robot deve continuare l'esplorazione se richiesto dall'utente.

  \item[\requirementref{R-blinkLed}]\labelsssec{R-blinkLed}

  Il LED del robot deve lampeggiare se esso è in fase di esplorazione.

  \item[\requirementref{R-consoleUpdate}]\labelsssec{R-consoleUpdate}

  La console deve contenere le informazioni aggiornate riguardo allo stato di esplorazione del robot.

  \item[\requirementref{R-stopAtBag}]\labelsssec{R-stopAtBag}

  Il robot deve fermarsi in prossimità di una valigia.

  \item[\requirementref{R-takePhoto}]\labelsssec{R-takePhoto}

  Il robot deve scattare una fotografia all'ostacolo appena individuato.

  \item[\requirementref{R-sendPhoto}]\labelsssec{R-sendPhoto}

  Il robot deve inviare la fotografia appena scattata alla console.

  \item[\requirementref{R-alert}]\labelsssec{R-alert}

  Il device dell'operatore, una volta ricevuta la foto, deve essere in grado di riconoscere
  se la valigia contiene una bomba, ed, in questo caso, inviare una notifica all'operatore.

  \item[\requirementref{R-storePhoto}]\labelsssec{R-storePhoto}

  La fotografia deve essere salvata in modo permanente contestualmente alle informazioni relative.

  \item[\requirementref{R-backHomeSinceBomb}]\labelsssec{R-backHomeSinceBomb}

  In caso la valigia contenga un ordigno, il robot Discovery deve tornare immediatamente alla posizione iniziale.

  \item[\requirementref{R-continueExploreAfterPhoto}]\labelsssec{R-continueExploreAfterPhoto}

  Se la valigia non contiene una bomba allora al robot viene comandato di continuare l'esplorazione, aggiornando contestualmente la mappa del territorio.
\end{description}

\subsection{Dettaglio sui requisiti della fase di recupero}\labelssec{retrieve}

\begin{description}[itemsep=1em]

  \item[\requirementref{R-waitForHome}]\labelsssec{R-waitForHome}

  Il sistema è in attesa che il robot discovery torni alla base.

  \item[\requirementref{R-reachBag}]\labelsssec{R-reachBag}

  Il robot Retriever deve raggiungere la posizione dell'ordigno.
  L'assunzione fatta è che il territorio sia rimasto invariato rispetto alla fase di esplorazione precedente.

  \item[\requirementref{R-bagAtHome}]\labelsssec{R-bagAtHome}

  Il robot Retriever dopo aver raccolto l'ordigno in un contenitore sicuro deve tornare alla base evitando gli ostacoli come in precedenza.

\end{description}
