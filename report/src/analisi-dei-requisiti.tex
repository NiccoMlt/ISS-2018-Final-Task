%% Direttive TeXworks:
% !TeX root = ../report.tex
% !TEX encoding = UTF-8 Unicode
% !TEX program = arara
% !TEX TS-program = arara
% !TeX spellcheck = it-IT

% arara: pdflatex: { synctex: yes, shell: yes, interaction: nonstopmode }
% arara: pdflatex: { synctex: yes, shell: yes, interaction: nonstopmode }
Dall'analisi dei requisiti forniti dal committente è possibile definire i seguenti requisiti:

\begin{enumerate}
  \item
    il sistema è distribuito e composto da tre componenti principali:
    \begin{description}
      \item[\textbf{Robot \requirementref{discovery}}] Si intende un componente che esplori (\requirementref{R-explore}) autonomamente l'ambiente seguendo specifici comandi dell'operatore;
      \item[\textbf{Robot \requirementref{retriever}}] Si intende un componente che raggiunga la bomba scoperta e la porti a casa;
      \item[\textbf{Componente \requirementref{console}}] Il componente utilizzato da remoto dall'operatore attraverso un PC od uno smartphone;
    \end{description}

  \item
    gli \textbf{ostacoli} sono \textit{statici} e, dato che viene assunto che la hall sia stata evacuata, possono essere divisi in:
    \begin{itemize}
      \item \textit{muri},
      \item \textit{borse}.
    \end{itemize}

  \item
    il \textbf{robot \requirementref{discovery}} deve iniziare l'esplorazione (\requirementref{R-explore}) solo nel caso in cui la temperatura sia sotto la soglia di 25°C (\requirementref{R-TempOk}) e riceva il comando di esplorazione dalla \textbf{\requirementref{console}} (\requirementref{R-startExplore});

  \item
    il comando \textit{Esplora} (\requirementref{R-startExplore}) viene inviato dalla \textbf{\requirementref{console}} unicamente al \textbf{robot \requirementref{discovery}};

  \item
    le informazioni dello stato del \textbf{robot \requirementref{discovery}} vengono inviate dallo stesso alla \textbf{\requirementref{console}} (\requirementref{R-consoleUpdate});
    \begin{description}
      \item[\textbf{Stato del robot}] Si intendono almeno le sue coordinate e l'azione corrente;
    \end{description}

  \item
    il \textbf{robot \requirementref{discovery}} è equipaggiato con un \textbf{sonar} nella sua parte anteriore per permettergli di individuare eventuali ostacoli (\requirementref{R-stopAtBag}) e di un \textbf{led} che dovrà saper far lampeggiare (\requirementref{R-blinkLed});

  \item
    qualora il \textbf{robot \requirementref{discovery}} incontri un ostacolo, deve:
    \begin{itemize}
      \item fermarsi (\requirementref{R-stopAtBag});
      \item scattare una foto (\requirementref{R-takePhoto});
      \item mandare la foto (\requirementref{R-sendPhoto}) alla \textbf{\requirementref{console}}.
    \end{itemize}
    A sua volta, la \textbf{\requirementref{console}} dovrà:
    \begin{itemize}
      \item allertare l'operatore (\requirementref{R-alert});
      \item salvare la foto (\requirementref{R-storePhoto}) ricevuta con dati di contesto;
      \item mandare il robot a casa (\requirementref{R-backHomeSinceBomb}) o farlo continuare (\requirementref{R-continueExploreAfterPhoto}).
    \end{itemize}

  \item
    il \textbf{robot \requirementref{retriever}} deve conoscere il percorso per raggiungere la bomba (\requirementref{R-reachBag}) una volta che il \textbf{robot \requirementref{discovery}} sarà tornato (\requirementref{R-waitForHome}) e il percorso per tornare con la bomba (\requirementref{R-bagAtHome});

  \item
    il cuore del sistema è la definizione della \textit{business logic} del robot che concerne l'esplorazione autonoma della hall;

  \item
    in qualsiasi momento, anche mentre in esplorazione, il \textbf{robot \requirementref{discovery}} deve essere reattivo ad un comando di halt (\requirementref{R-stopExplore}) inviatogli dalla \textbf{\requirementref{console}}.
\end{enumerate}
