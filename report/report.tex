%% Direttive TeXworks:
% !TeX root = ./report.tex
% !TEX encoding = UTF-8 Unicode
% !TEX program = arara
% !TEX TS-program = arara
% !TeX spellcheck = it-IT

% arara: pdflatex: { synctex: yes, shell: yes }
% arara: pdflatex: { synctex: yes, shell: yes }

%% Genera un file report.xmpdata con i dati di titolo e autore per il formato PDF/A %%
\begin{filecontents*}{\jobname.xmpdata}
  \Title{Laboratorio~di~Sistemi~Software~--~Tema~finale}
  \Author{Niccolò Maltoni}
  \Copyright{Questo documento è fornito sotto licenza Creative Commons Attribution-ShareAlike 4.0 International}
  \CopyrightURL{http://creativecommons.org/licenses/by-sa/4.0}
\end{filecontents*}

\documentclass{llncs}           % LaTeX document class for Lecture Notes in Computer Science

%%%%%%%%%%%%%%%%%%%%%%%%%%%%%%%%%%%%%%%%%%%%%%%%%%%%%%%%%%%
%% package sillabazione italiana e uso lettere accentate
\usepackage[T1]{fontenc}        % serve per impostare la codifica di output del font
\usepackage{textcomp}           % serve per fornire supporto ai Text Companion fonts
\usepackage[utf8]{inputenc}     % serve per impostare la codifica di input del font
\usepackage[
  english,            % utilizza l'inglese come lingua secondaria
  italian             % utilizza l'italiano come lingua primaria
]{%
  babel,                      % serve per scrivere Indice, Capitolo, etc in Italiano
  varioref                    % introduce il comando \vref da usarsi nello stesso modo del comune \ref per i riferimenti
}
\usepackage{lmodern}            % carica una variante Latin Modern prodotto dal GUST
\usepackage[%
    strict,             % rende tutti gli warning degli errori
    autostyle,          % imposta lo stile in base al linguaggio specificato in babel
    english=american,   % imposta lo stile per l'inglese
    italian=guillemets  % imposta lo stile per l'italiano
]{csquotes}                     % serve a impostare lo stile delle virgolette
%%%%%%%%%%%%%%%%%%%%%%%%%%%%%%%%%%%%%%%%%%%%%%%%%%%%%%%%%%%%%

\usepackage{indentfirst}      % serve per avere l'indentazione nel primo paragrafo
\usepackage{url}
\usepackage{setspace}         % serve a fornire comandi di interlinea standard
\usepackage{xspace}

\makeatletter

%%%%%%%%%%%%%%%%%%%%%%%%%%%%%% User specified LaTeX commands.
\usepackage{manifest}

\makeatother

%%%%%%%
\newif\ifpdf{}
\ifx\pdfoutput\undefined{}
  \pdffalse{}         % we are not running PDFLaTeX
\else
  \pdfoutput=1        % we are running PDFLaTeX
  \pdftrue{}
\fi
%%%%%%%

\usepackage{xcolor}             % serve per la gestione dei colori nel testo

%%%%%%%%%%%%%%%
\ifpdf{}
  \usepackage[pdftex]{graphicx} % serve per includere immagini e grafici
\else
  \usepackage{graphicx}         % serve per includere immagini e grafici
\fi
%%%%%%%%%%%%%%%
\ifpdf{}
  \DeclareGraphicsExtensions{.pdf, .jpg, .tif, .png} % chktex 26
\else
  \DeclareGraphicsExtensions{.eps, .jpg, .png} % chktex 26
\fi
%%%%%%%%%%%%%%%

\newcommand{\java}{\textsf{Java}}
\newcommand{\contact}{\emph{Contact}}
\newcommand{\corecl}{\texttt{corecl}}
\newcommand{\medcl}{\texttt{medcl}}
\newcommand{\msgcl}{\texttt{msgcl}}
\newcommand{\android}{\texttt{Android}}
\newcommand{\dsl}{\texttt{DSL}}
\newcommand{\jazz}{\texttt{Jazz}}
\newcommand{\rtc}{\texttt{RTC}}
\newcommand{\ide}{\texttt{Contact-ide}}
\newcommand{\xtext}{\texttt{XText}}
\newcommand{\xpand}{\texttt{Xpand}}
\newcommand{\xtend}{\texttt{Xtend}}
\newcommand{\pojo}{\texttt{POJO}}
\newcommand{\junit}{\texttt{JUnit}}

\newcommand{\action}[1]{\texttt{#1}\xspace}
\newcommand{\code}[1]{{\small{\texttt{#1}}}\xspace}
\newcommand{\codescript}[1]{{\scriptsize{\texttt{#1}}}\xspace}

% Cross-referencing
\newcommand{\labelsec}[1]{\label{sec:#1}}
\newcommand{\xs}[1]{\sectionname~\ref{sec:#1}}
\newcommand{\xsp}[1]{\sectionname~\ref{sec:#1} \onpagename~\pageref{sec:#1}}
\newcommand{\labelssec}[1]{\label{ssec:#1}}
\newcommand{\xss}[1]{\subsectionname~\ref{ssec:#1}}
\newcommand{\xssp}[1]{\subsectionname~\ref{ssec:#1} \onpagename~\pageref{ssec:#1}}
\newcommand{\labelsssec}[1]{\label{sssec:#1}}
\newcommand{\xsss}[1]{\subsectionname~\ref{sssec:#1}}
\newcommand{\xsssp}[1]{\subsectionname~\ref{sssec:#1} \onpagename~\pageref{sssec:#1}}
\newcommand{\labelfig}[1]{\label{fig:#1}}
\newcommand{\xf}[1]{\figurename~\ref{fig:#1}}
\newcommand{\xfp}[1]{\figurename~\ref{fig:#1} \onpagename~\pageref{fig:#1}}
\newcommand{\labeltab}[1]{\label{tab:#1}}
\newcommand{\xt}[1]{\tablename~\ref{tab:#1}}
\newcommand{\xtp}[1]{\tablename~\ref{tab:#1} \onpagename~\pageref{tab:#1}}
% Category Names
\newcommand{\sectionname}{Section}
\newcommand{\subsectionname}{Subsection}
\newcommand{\sectionsname}{Sections}
\newcommand{\subsectionsname}{Subsections}
\newcommand{\secname}{\sectionname}
\newcommand{\ssecname}{\subsectionname}
\newcommand{\secsname}{\sectionsname}
\newcommand{\ssecsname}{\subsectionsname}
\newcommand{\onpagename}{on page}

\newcommand{\xauthA}{Niccolò~Maltoni}
\newcommand{\xfaculty}{II~Faculty~of~Engineering}
\newcommand{\xunibo}{Alma~Mater~Studiorum~--~University~of~Bologna}

\setcounter{secnumdepth}{3}   % Numera fino alla sottosezione nel corpo del testo
\setcounter{tocdepth}{3}      % Numera fino alla sotto-sottosezione nell'indice

\usepackage[%
  depth=3,              % equivale a bookmarksdepth di hyperref
  open=false,           % equivale a bookmarksopen di hyperref
  numbered=true         % equivale a bookmarksnumbered di hyperref
]{bookmark}                     % Gestisce i segnalibri meglio di hyperref
\usepackage{hyperref}           % Gestisce tutte le cose ipertestuali del pdf
\hypersetup{%
  pdfpagemode={UseNone},
  hidelinks,            % nasconde i collegamenti (non vengono quadrettati)
  hypertexnames=false,
  linktoc=all,          % inserisce i link nell'indice
  unicode=true,         % only Latin characters in Acrobat’s bookmarks
  pdftoolbar=false,     % show Acrobat’s toolbar?
  pdfmenubar=false,     % show Acrobat’s menu?
  plainpages=false,
  breaklinks,
  pdfstartview={Fit},
  pdfauthor={Niccolò Maltoni},
  pdfcreator={Niccolò Maltoni},
  pdftitle={Laboratorio di Sistemi Software - Tema finale}, % chktex 8
  pdflang={it}
}

\usepackage[a-1b]{pdfx}
\usepackage[%
  italian,            % definizione delle lingue da usare
  nameinlink          % inserisce i link nei riferimenti
]{cleveref}                     % permette di usare riferimenti migliori dei \ref e dei varioref

\begin{document}

\title{Software Engineering\\
 process report template}

\author{\xauthA}

\institute{%
  \xunibo{}\\\email{}\ niccolo.maltoni@studio.unibo.it
}

\maketitle

%===========================================================================
\section{Introduction}
\labelsec{intro}
%===========================================================================

%===========================================================================
\section{Vision}
\labelsec{Vision}
%===========================================================================

%===========================================================================
\section{Goals}
\labelsec{Goals}
%===========================================================================

%===========================================================================
\section{Requirements}
\labelsec{Requirements}
%===========================================================================

 
%===========================================================================
\section{Requirement analysis}
\labelsec{ReqAnalysis}
%===========================================================================
\subsection{Use cases}
\labelssec{UseCases}

\subsection{Scenarios}
\labelssec{Scenarios}

\subsection{(Domain) model}

\subsection{Test plan}

%===========================================================================
\section{Problem analysis}
\labelsec{ProblemAnalysis}
%===========================================================================
\subsection{Logic architecture}
\subsection{Abstraction gap}
\subsection{Risk analysis}

%===========================================================================
\section{Work plan}
\labelsec{wplan}
%===========================================================================

%===========================================================================
\section{Project}
\labelsec{Project}
%===========================================================================

\subsection{Structure}
\subsection{Interaction}
\subsection{Behavior}

%===========================================================================
\section{Implementation}
\labelsec{Implementation}
%===========================================================================

%===========================================================================
\section{Testing}
\labelsec{testing}
%===========================================================================

%===========================================================================
\section{Deployment}
\labelsec{Deployment}
%===========================================================================

%===========================================================================
\section{Maintenance}
\labelsec{Maintenance}
%===========================================================================

\appendix

\bibliographystyle{abbrv}
\bibliography{biblio}

\end{document}
