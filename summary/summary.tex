%% Direttive TeXworks:
% !TeX root = ./summary.tex
% !TeX encoding = UTF-8 Unicode
% !TeX spellcheck = it_IT
% !TeX program = arara
% !TeX options = --log --verbose --language=it "%DOC%"

% arara: pdflatex:      { interaction: nonstopmode, shell: yes }
% arara: pdflatex:      { interaction: nonstopmode, shell: yes }
% arara: pdflatex:      { interaction: nonstopmode, synctex: yes, shell: yes }

\documentclass{llncs}

\usepackage{a4wide}

%%%%%%%%%%%%%%%%%%%%%%%%%%%%%%%%%%%%%%%%%%%%%%%%%%%%%%%%%%%
%% package sillabazione italiana e uso lettere accentate
\usepackage[T1]{fontenc}
\usepackage{textcomp}
\usepackage[utf8]{inputenc}
\usepackage[english,italian]{babel,varioref}
\usepackage{lmodern}
\usepackage[%
  strict,
  autostyle,
  english=american,
  italian=guillemets
]{csquotes}
%%%%%%%%%%%%%%%%%%%%%%%%%%%%%%%%%%%%%%%%%%%%%%%%%%%%%%%%%%%%%

\usepackage{fancyvrb}

%% Genera un file report.xmpdata con i dati di titolo e autore per il formato PDF/A %%
\begin{VerbatimOut}{\jobname.xmpdata}
  \Title{Laboratorio di Sistemi Software --- Breve presentazione del tema finale}
  \Subject{Realizzazione di un sistema per l'individuazione e rimozione di una bomba dalla hall di un aeroporto tramite robot controllato da remoto.}
  \Author{Niccolò Maltoni}
  \Copyright{Questo documento è fornito sotto licenza Apache License, Version 2.0}
  \CopyrightURL{https://opensource.org/licenses/Apache-2.0}
\end{VerbatimOut}

\usepackage{relsize, etoolbox}
\AtBeginEnvironment{foreigndisplayquote}{\smaller}

\usepackage{indentfirst}
\usepackage{xurl}
\usepackage{setspace}
\usepackage{xspace}

\makeatletter

%%%%%%%%%%%%%%%%%%%%%%%%%%%%%% User specified LaTeX commands.
\usepackage{manifest}

\makeatother

\usepackage{natbib}
\usepackage{xcolor}

\usepackage{graphicx}
\DeclareGraphicsExtensions{.eps, .pdf, .jpg, .tif, .png} % chktex 26

\usepackage{subcaption}
\usepackage{float}

\usepackage[savemem]{listings}
\usepackage{listingsutf8}

\definecolor{dkgreen}{rgb}{0,0.6,0}
\definecolor{gray}{rgb}{0.5,0.5,0.5}
\definecolor{mauve}{rgb}{0.58,0,0.82}

\lstset{
  extendedchars=true,
  inputencoding=utf8/latin1,
  frame=single,
  captionpos=b,
  language=Java,
  showspaces=false,
  showtabs=false,
  showstringspaces=false,
  columns=flexible,
  basicstyle={\small\ttfamily},
  numbers=none,
  numberstyle=\tiny\color{gray},
  keywordstyle=\color{blue},
  commentstyle=\color{dkgreen},
  stringstyle=\color{mauve},
  breaklines=true,
  breakatwhitespace=true,
  keepspaces=true,
  numbersep=5pt,
  tabsize=2
}

\lstdefinelanguage{qa}{
  basicstyle=\ttfamily\scriptsize,
  numbers=left,
  numberstyle=\scriptsize,
  stepnumber=1,
  numbersep=8pt,
  tabsize=2,
  showstringspaces=false,
  breaklines=true,
  breakatwhitespace=true,
  keywordstyle=\color{mauve}\bfseries,
  commentstyle=\color{dkgreen},stringstyle=\color{blue},
  morekeywords={System,Event,Dispatch,Context,QActor,Rules,State,%
      demo,whenMsg,whenEvent,onMsg,onEvent,transition,stopAfter,%
      whenTime,resumeLastPlan,initial,finally,javaRun,println,%
      emit,forward,do},
  otherkeywords={:,->,-m,;,.,\,,[,],:-},
  morestring=*[d]{"}, % chktex 18
  morecomment=[l]{//},
  morecomment=[s]{/*}{*/}
}

\newcommand{\java}{\textsf{Java}}
\newcommand{\contact}{\emph{Contact}}
\newcommand{\corecl}{\texttt{corecl}}
\newcommand{\medcl}{\texttt{medcl}}
\newcommand{\msgcl}{\texttt{msgcl}}
\newcommand{\android}{\texttt{Android}}
\newcommand{\dsl}{\texttt{DSL}}
\newcommand{\jazz}{\texttt{Jazz}}
\newcommand{\rtc}{\texttt{RTC}}
\newcommand{\ide}{\texttt{Contact-ide}}
\newcommand{\xtext}{\texttt{XText}}
\newcommand{\xpand}{\texttt{Xpand}}
\newcommand{\xtend}{\texttt{Xtend}}
\newcommand{\pojo}{\texttt{POJO}}
\newcommand{\junit}{\texttt{JUnit}}

\newcommand{\action}[1]{\texttt{#1}\xspace}
\newcommand{\code}[1]{{\small{\texttt{#1}}}\xspace}
\newcommand{\codescript}[1]{{\scriptsize{\texttt{#1}}}\xspace}

\newcommand{\requirement}[1]{\hypertarget{req:#1}{\textcolor{blue}{#1}}}
\newcommand{\requirementref}[1]{\hyperlink{req:#1}{\textcolor{blue}{#1}}}

% Cross-referencing
\newcommand{\labelsec}[1]{\label{sec:#1}}
\newcommand{\xs}[1]{\sectionname~\ref{sec:#1}}
\newcommand{\xsp}[1]{\sectionname~\ref{sec:#1} \onpagename~\pageref{sec:#1}}
\newcommand{\labelssec}[1]{\label{ssec:#1}}
\newcommand{\xss}[1]{\subsectionname~\ref{ssec:#1}}
\newcommand{\xssp}[1]{\subsectionname~\ref{ssec:#1} \onpagename~\pageref{ssec:#1}}
\newcommand{\labelsssec}[1]{\label{sssec:#1}}
\newcommand{\xsss}[1]{\subsectionname~\ref{sssec:#1}}
\newcommand{\xsssp}[1]{\subsectionname~\ref{sssec:#1} \onpagename~\pageref{sssec:#1}}
\newcommand{\labelfig}[1]{\label{fig:#1}}
\newcommand{\xf}[1]{\figurename~\ref{fig:#1}}
\newcommand{\xfp}[1]{\figurename~\ref{fig:#1} \onpagename~\pageref{fig:#1}}
\newcommand{\labeltab}[1]{\label{tab:#1}}
\newcommand{\xt}[1]{\tablename~\ref{tab:#1}}
\newcommand{\xtp}[1]{\tablename~\ref{tab:#1} \onpagename~\pageref{tab:#1}}
% Category Names
\newcommand{\sectionname}{Section}
\newcommand{\subsectionname}{Subsection}
\newcommand{\sectionsname}{Sections}
\newcommand{\subsectionsname}{Subsections}
\newcommand{\secname}{\sectionname}
\newcommand{\ssecname}{\subsectionname}
\newcommand{\secsname}{\sectionsname}
\newcommand{\ssecsname}{\subsectionsname}
\newcommand{\onpagename}{on page}

\newcommand{\xauthA}{Niccolò~Maltoni}
\newcommand{\xfaculty}{II~Faculty~of~Engineering}
\newcommand{\xunibo}{Alma~Mater~Studiorum~--~University~of~Bologna}

\usepackage{enumitem}
\setlist[itemize]{itemsep=1em,topsep=1em,itemindent=0.5\parindent}
\setlist[enumerate]{itemsep=1em,topsep=1em,itemindent=0.5\parindent}
\setlist[description]{itemsep=1em,topsep=1em,itemindent=0.25\parindent}

\setcounter{secnumdepth}{2}
\setcounter{tocdepth}{3}

\usepackage[a-1b]{pdfx}

\usepackage[depth=3,open=false,numbered=true]{bookmark}

\hypersetup{%
  pdfpagemode={UseNone},
  hidelinks,
  hypertexnames=false,
  linktoc=all,
  unicode=true,
  pdftoolbar=false,
  pdfmenubar=false,
  plainpages=false,
  breaklinks,
  pdfstartview={Fit},
  pdflang={it}
}

\usepackage[italian,nameinlink]{cleveref}

\begin{document}

\title{%
  \textbf{LSS --- Laboratorio di Sistemi Software}\\%
  Breve presentazione del tema finale\\%
  \textit{\small \url{https://github.com/NiccoMlt/ISS-2018-Final-Task}}
}

\author{\xauthA}

\institute{%
  \xunibo{}\\\email{}\ \href{mailto:niccolo.maltoni@studio.unibo.it}{niccolo.maltoni@studio.unibo.it}
}

{\def\addcontentsline#1#2#3{}\maketitle}

\tableofcontents

%===========================================================================
\section{Introduzione}\labelsec{intro}

TODO % TODO

%===========================================================================

%===========================================================================
% \section{Visione}\labelsec{vision}
%===========================================================================

%===========================================================================
% \section{Obiettivi}\labelsec{goals}
%===========================================================================

%===========================================================================
% \section{Requisiti}\labelsec{requirements}
%===========================================================================

%===========================================================================
\section{Analisi dei requisiti}\labelsec{req_analysis}
TODO % TODO
%===========================================================================
% \subsection{Casi d'uso}\labelssec{use_cases}

% \subsection{Scenario}\labelssec{scenarios}

% \subsection{Modello di dominio}\labelssec{modello}

% \subsection{Test plan}\labelssec{test_plan}

%===========================================================================
% \section{Analisi del problema}\labelsec{problem_analysis}

%===========================================================================
% \subsection{Architettura logica}\labelssec{logic_arch}

% \subsection{Abstraction gap}\labelssec{abstraction-gap}

% \subsection{Analisi del rischio}\labelssec{risk_analysis}

%===========================================================================
% \section{Piano di lavoro}\labelsec{work_plan}

%===========================================================================

%===========================================================================
\section{Progetto}\labelsec{project}
TODO % TODO
%===========================================================================

% \subsection{Struttura}\labelssec{structure}
% \subsection{Interazione}\labelssec{interaction}
% \subsection{Comportamento}\labelssec{behavior}

%===========================================================================
% \section{Implementazione}\labelsec{implementation}

%===========================================================================

%===========================================================================
% \section{Testing}\labelsec{testing}

%===========================================================================

%===========================================================================
% \section{Deployment}\labelsec{deployment}
%===========================================================================

%===========================================================================
% \section{Manutenzione}\labelsec{maintenance}

%===========================================================================

\newpage

% \section{Difficoltà e note varie}

Durante lo sviluppo, sono state riscontrate diverse difficoltà che per completezza è bene elencare.

Risulta comunque evidente che queste criticità rispecchino il mondo reale e dunque bene si vadano ad integrare con lo scopo di questo progetto di rispecchiare fedelmente un processo di sviluppo all'interno di una software house.

\subsection{Creazione dinamica degli attori}
Sarebbe stato molto utile poter creare dinamicamente gli attori in QA\@:
infatti, sarebbe stato possibile creare un robot adapter per ogni robot a runtime senza dover assegnare tutti i robot usati allo stesso adapter.

Per esempio, sarebbe stato possibile usare un robot per l'esplorazione ed uno differente per il recupero.

\subsection{Tecnologie datate}
Esperienza interessante è stata il dover avere a che fare con tecnologie almeno in parte datate o non complete:

\begin{itemize}
  \item
    come specificato più nel dettaglio in~\Cref{app:raspi},
    l'immagine per Raspberry Pi fornita come riferimento dalla software house è basata su una versione datata dei pacchetti, e la configurazione in parte fuori standard ha convinto a realizzare una nuova immagine al solo scopo di questo progetto.
  \item
    il linguaggio QA è eccezionale sotto il punto di vista della realizzazione di metamodelli e per il \textit{bootstrap} di prototipi, ma come anche da documentazione ci sono alcune funzionalità solo teorizzate o implementate in parte che possono risultare limitanti dal lato pratico.
    \begin{itemize}
      \item
        è comunque importante specificare che è stata usata la versione 1.5.13.2, in quanto quella presentata durante le lezioni e meglio conosciuta;
        è probabile che la nuova versione disponibile possa arginare alcune delle criticità delle versioni più vecchie.
    \end{itemize}
  \item
    la software factory richiede necessariamente Eclipse per la code generation e presenta dunque un lock-in non indifferente rispetto a tecnologie differenti basate su build system (ad esempio, Gradle) anziché su IDE, anche in ottica di continuous integration.
  \item
    il template fornito per realizzare questo report è basato su una versione datata e modificata della classe \LaTeX{} \texttt{LLNCS} fornita da Springer\footnote{\url{https://www.springer.com/gp/computer-science/lncs/conference-proceedings-guidelines}};
    a causa di questo vincolo, sono state riscontrate difficoltà con l'utilizzo di pacchetti moderni (come ad esempio \texttt{subcaption}).
\end{itemize}

\subsection{Modularizzazione del metamodello QA}
Durante lo sviluppo è stata riscontrata una limitazione non indifferente, soprattutto in termini di sintesi e pulizia del codice:
i file QA prodotti tendono a diventare molto grandi (anche più di 700 righe) e diventano di difficile fruizione.

\subsection{Stampa degli errori}
Durante lo sviluppo, un grosso rallentamento è stato causato dalla mancata stampa su console di errori incontrati, poiché ciò rende molto più complesso il debug.

\subsection{MQTT non funzionante}
Durante lo sviluppo sono stati riscontrati diversi problemi con l'uso di MQTT\@:
infatti, la software factory incita l'utilizzo di MQTT per la realizzazione della comunicazione tra le entità in gioco, ma ha grossi problemi di implementazione
\begin{itemize}
    \item pacchetti persi;
    \item problemi durante l'avvio degli attori dovuti ad una sovrapposizione delle connessioni;
    \item ritardo di comunicazione anche su singola macchina in comunicazione tramite localhost.
\end{itemize}
 % TODO

% %% Direttive TeXworks:
% !TeX root = ../report.tex
% !TEX encoding = UTF-8 Unicode
% !TeX spellcheck = it-IT

\begin{appendix}
  \section{Appendice: Immagine del Raspberry Pi}\label{app:raspi}

  La software house metteva a disposizione una immagine della distribuzione Raspbian\footnote{\url{https://www.raspberrypi.org/downloads/raspbian/}}
  (una versione di Debian per Raspberry Pi) adattata dalla software house per l'uso interno.

  L'immagine è però basata su una vecchia versione di Raspbian Jessie datata 26 febbraio 2016 e presenta diversi problemi.
  Già in passato la software house aveva riscontrato problemi e aveva proposto \textit{fix}
  documentati sul sito\footnote{\url{http://htmlpreview.github.io/?https://github.com/anatali/iss2018/blob/master/it.unibo.issMaterial/issdocs/Material/LectureCesena1819.html}}
  della software house.

  La configurazione standard prevede l'accesso alla rete pubblica tramite WiFi e la configurazione locale tramite connessione diretta ethernet;
  poiché la macchina di sviluppo utilizzata non dispone di connessione ethernet, si è deciso,
  consci anche degli altri problemi presenti nell'immagine, di adattare una versione più moderna di Raspbian Buster lite per questo progetto.

  \subsection{Configurazione del Raspberry Pi}\label{app:raspi:conf}

  Si è scelto di utilizzare un modulo WiFi USB in aggiunta a quello integrato nel Raspberry Pi 3,
  ma è possibile utilizzare il solo modulo WiFi integrato sia come access point che come client, pur con maggiori instabilità e minori velocità.

  La procedura di adattamento dell'immagine è stata documentata in modo da poter essere riproducibile in futuro.

  \begin{enumerate}
    \item
      La base dell'immagine è l'ultima versione di \textbf{Raspbian} nella sua versione minimale \textbf{lite}:
      in questo caso, la versione Buster del 26 settembre 2019;
    \item
      l'immagine è stata connessa alla rete tramite WiFi e i pacchetti sono stati aggiornati tramite \texttt{apt};
    \item
      seguendo la documentazione fornita sul sito ufficiale\footnote{\url{https://raspap.com/}},
      è stato installato \textbf{RaspAP} e configurato per funzionare esclusivamente via WiFi:
      \begin{enumerate}
        \item
          \texttt{curl -sL https://install.raspap.com | bash} permette di installare RaspAP\@;
        \item
          seguendo le FAQ\footnote{\url{https://github.com/billz/raspap-webgui/wiki/FAQs\#can-i-use-wlan0-and-wlan1-rather-than-eth0-for-my-ap}}
          al file \texttt{includes/config.php} si è aggiunta \texttt{wlan1} come client interface e si è impostato nel file \texttt{/etc/dhcpcd.conf}
          \texttt{wlan0} come access point
      \end{enumerate}
      In questo modo, il dispositivo dovrebbe essere in grado di collegarsi alla rete WiFi pre-configurata e mettere a disposizione una rete con SSID ``raspi-webgui''.
    \item
      utilizzando lo script ufficiale\footnote{\url{https://github.com/docker/docker-install}}
      (ma si sarebbe potuta seguire la documentazione\footnote{\url{https://docs.docker.com/install/linux/docker-ce/debian/}}), si è installato Docker;
    \item
      si è installato Java 1.8 tramite pacchetto \texttt{openjdk-8-jdk-headless};
    \item
      il broker MQTT Mosquitto viene eseguito tramite container Docker\footnote{\url{https://hub.docker.com/_/eclipse-mosquitto}}.
  \end{enumerate}
\end{appendix}
 % TODO

\end{document}
